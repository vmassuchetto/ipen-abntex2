%% Copyright 2012-2014 by abnTeX2 group at http://abntex2.googlecode.com/
%%
%% This work may be distributed and/or modified under the
%% conditions of the LaTeX Project Public License, either version 1.3
%% of this license or (at your option) any later version.
%% The latest version of this license is in
%%   http://www.latex-project.org/lppl.txt
%% and version 1.3 or later is part of all distributions of LaTeX
%% version 2005/12/01 or later.
%%
%% This work has the LPPL maintenance status `maintained'.
%%
%% The Current Maintainer of this work is the abnTeX2 team, led
%% by Lauro César Araujo. Further information are available on
%% http://abntex2.googlecode.com/
%%
%% This work consists of the files iptex.sty and exemplo.tex, and implements
%% the documentation standard for the Nuclear Energy Research Institute
%% (IPEN) education programs.

% Classe do documento
% -------------------

\documentclass[
	12pt,                       % tamanho da fonte
	oneside,                    % impressão em um lado
	a4paper,                    % tamanho do papel.
	chapter=TITLE,              % títulos de capítulos em letras maiúsculas
	section=TITLE,              % títulos de seções em letras maiúsculas
	brazil                      % o último idioma é o principal do documento
    sumario=tradicional         % utiliza o sumário comum do abnTeX2
]{abntex2}

% Pacotes utilizados
% ------------------

\usepackage{helvet}                             % fonte helveltica
\usepackage[T1]{fontenc}                        % seleçã de códigos de fonte
\usepackage[utf8]{inputenc}                     % codificação UTF-8
\usepackage{lastpage}                           % usado pela ficha catalográfica
\usepackage{indentfirst}                        % denteia o primeiro parágrafo de cada seção
\usepackage{color}                              % controle das cores
\usepackage{graphicx}                           % inclusão de figuras
\usepackage{geometry}                           % espaçamentos de página
\usepackage{microtype}                          % melhorias da justificação
\usepackage{lipsum}                             % gera textos de exemplo
\usepackage[alf]{abntex2cite}	                % citações no padrão ABNT
\usepackage{ipen}                               % formatação da capa e da folha de rosto

% Configurações dos pacotes
% -------------------------

% Fonte arial como padrão

\renewcommand{\familydefault}{\sfdefault}

% Capa e folha de rosto

\titulo{Título do Trabalho}
\subtitulo{Subtítulo do trabalho}               % deixe como \subtitulo{\ } se não quiser usar
\autor{Seu Nome}
\instituicao{Instituto de Pesquisas Energéticas e Nucleares}
\instituicaosub{Autarquia associada à Universidade de São Paulo}
\local{São Paulo}
\data{2014}
\orientador{Nome do Orientador}
\preambulo{(Dissertação ou Tese) apresentada como parte dos requisitos para
  obtenção do Grau de (Mestre ou Doutor) em Ciências na Área de Tecnologia
  Nuclear – (Aplicações ou Materiais ou Reatores);}

% Informações do PDF

\makeatletter
\hypersetup{
  pdftitle={\@title},
  pdfauthor={\@author},
  pdfsubject={\imprimirpreambulo},
  pdfcreator={LaTeX with abnTeX2},
  pdfkeywords={palavras}{chave}{aqui},
  colorlinks=false,
  hidelinks=true
}
\makeatother

\setlrmarginsandblock{3.5cm}{2.0cm}{*}  % margens direita e esquerda
\setulmarginsandblock{3.0cm}{2.0cm}{*}  % margens superior e inferior
\checkandfixthelayout
\setlength{\parindent}{2.0cm}           % recuo do parágrafo
\setlength{\parskip}{0.2cm}             % espaçamento entre parágrafos
\linespread{1.3}                        % espaçamento padrão de 1,5 entre linhas
\makeindex                              % índice

% Início do documento
% -------------------

\begin{document}
\frenchspacing                  % retira espaços extras entre as frases
\imprimircapa                   % capa
\imprimirfolhaderosto           % folha de rosto

% Ficha bibliografica
%
% Isto é um exemplo de Ficha Catalográfica, ou ``Dados internacionais de
% catalogação-na-publicação''. Você pode utilizar este modelo como referência.
% Porém, provavelmente a biblioteca da sua universidade lhe fornecerá um PDF
% com a ficha catalográfica definitiva após a defesa do trabalho. Quando estiver
% com o documento, salve-o como PDF no diretório do seu projeto e substitua todo
% o conteúdo de implementação deste arquivo pelo comando abaixo:
%
% \begin{fichacatalografica}
%   \includepdf{ficha-catalografica.pdf}
% \end{fichacatalografica}
%
% \begin{fichacatalografica}
%   \vspace*{\fill}					% posição vertical
%   \hrule							% linha horizontal
%   \begin{center}					% minipage centralizada
%   \begin{minipage}[c]{12.5cm}		% largura
% 	  \imprimirautor
% 	  \hspace{0.5cm}
%     \imprimirtitulo / \imprimirautor. -- \imprimirlocal, \imprimirdata
%     \hspace{0.5cm} \pageref{LastPage} p. : il. (algumas color.) ; 30 cm.\\
%     \hspace{0.5cm} \imprimirorientadorRotulo~\imprimirorientador\\
%     \hspace{0.5cm}
%     \parbox[t]{\textwidth}{\imprimirtipotrabalho~--~\imprimirinstituicao,
%     \imprimirdata.}\\
% 	  \hspace{0.5cm}
% 		1. Palavra-chave1.
% 		2. Palavra-chave2.
% 		I. Orientador.
% 		II. Universidade xxx.
% 		III. Faculdade de xxx.
% 		IV. Título\\
% 	  \hspace{8.75cm} CDU 02:141:005.7\\
% 	\end{minipage}
%   \end{center}
% \hrule
% \end{fichacatalografica}
% ---

% Dedicatória

\begin{dedicatoria}
  \vspace*{\fill}
  \centering
  \noindent
  \textit{ Este trabalho é dedicado às crianças adultas que,\\
  quando pequenas, sonharam em se tornar cientistas.} \vspace*{\fill}
\end{dedicatoria}

% Agradecimentos

\begin{agradecimentos}
  Os agradecimentos principais são direcionados à Gerald Weber, Miguel Frasson,
  Leslie H. Watter, Bruno Parente Lima, Flávio de Vasconcellos Corrêa, Otavio Real
  Salvador, Renato Machnievscz\footnote{Os nomes dos integrantes do primeiro
  projeto abn\TeX\ foram extraídos de
  \url{http://codigolivre.org.br/projects/abntex/}} e todos aqueles que
  contribuíram para que a produção de trabalhos acadêmicos conforme
  as normas ABNT com \LaTeX\ fosse possível.

  Agradecimentos especiais são direcionados ao Centro de Pesquisa em Arquitetura
  da Informação\footnote{\url{http://www.cpai.unb.br/}} da Universidade de
  Brasília (CPAI), ao grupo de usuários
  \emph{latex-br}\footnote{\url{http://groups.google.com/group/latex-br}} e aos
  novos voluntários do grupo
  \emph{\abnTeX}\footnote{\url{http://groups.google.com/group/abntex2} e
  \url{http://abntex2.googlecode.com/}}~que contribuíram e que ainda
  contribuirão para a evolução do \abnTeX.

\end{agradecimentos}

% Epígrafe

\begin{epigrafe}
  \vspace*{\fill}
  \begin{flushright}
    \textit{``Não vos amoldeis às estruturas deste mundo, \\
 	mas transformai-vos pela renovação da mente, \\
 	a fim de distinguir qual é a vontade de Deus: \\
 	o que é bom, o que Lhe é agradável, o que é perfeito.\\
 	(Bíblia Sagrada, Romanos 12, 2)}
  \end{flushright}
\end{epigrafe}

% Resumos

\newgeometry{top=5cm,left=3.5cm,right=2cm,bottom=2cm}
\begin{resumo}
  Segundo a NBR6028:2003, o resumo deve ressaltar o objetivo, o método, os
  resultados e as conclusões do documento. A ordem e a extensão destes itens
  dependem do tipo de resumo (informativo ou indicativo) e do tratamento que
  cada item recebe no documento original. O resumo deve ser precedido da
  referência do documento, com exceção do resumo inserido no próprio documento.
  (\ldots) As palavras-chave devem figurar logo abaixo do resumo, antecedidas
  da expressão Palavras-chave:, separadas entre si por ponto e finalizadas
  também por ponto.

  \vspace{\onelineskip}
  \noindent
  \textbf{Palavras-chaves}: latex, abntex, editoração de texto
\end{resumo}

\begin{resumo}[Abstract]
  \begin{otherlanguage*}{english}
    This is the english abstract. \lipsum[20]

    \vspace{\onelineskip}
    \noindent
    \textbf{Keywords}: latex, abntex, text editoration
  \end{otherlanguage*}
\end{resumo}
\restoregeometry

% Lista de ilustrações

\pdfbookmark[0]{\listfigurename}{lof}
\listoffigures*
\cleardoublepage

% Lista de tabelas

\pdfbookmark[0]{\listtablename}{lot}
\listoftables*
\cleardoublepage

% Lista de abreviaturas e siglas

\begin{siglas}
  \item[ABNT] Associação Brasileira de Normas Técnicas
  \item[abnTeX] ABsurdas Normas para TeX
\end{siglas}

% Lista de símbolos

\begin{simbolos}
  \item[$ \Gamma $] Letra grega Gama
  \item[$ \Lambda $] Lambda
  \item[$ \zeta $] Letra grega minúscula zeta
  \item[$ \in $] Pertence
\end{simbolos}

% Sumário

\pdfbookmark[0]{\contentsname}{toc}
\addcontentsline{toc}{chapter}{Introdução}
\addcontentsline{toc}{chapter}{Conclusão}
\tableofcontents*
\cleardoublepage

\textual                % elementos textuais
\pagestyle{simple}      % numeração na página do cabeçalho

\chapter*{Introdução}

Este modelo de documento foi construído com base no guia de teses da pós
graduação do IPEN. Os títulos dos capítulos devem estar em maiúscula e negrito.
Chamadas para referências podem ser feitas dentro do texto assim:
\citeonline{Adelman1995}, ou no final de um parágrafo assim \cite{Adelman1995}.
Na verdade só querermos colocar uma referência aqui para aparecer a seção de
referências no final.

\lipsum[50]

\chapter{Primeiro capítulo}

A figura \ref{Fig:Logomarca} é um exemplo para que tenhamos a lista de figuras. \lipsum[20]

\begin{figure}[htb]
  \caption{Logomarca do IPEN}
  \begin{center}
	\includegraphics[width=0.6\textwidth]{./ipen.png}
  \end{center}
  \legend{Fonte: Internet}
  \label{Fig:Logomarca}
\end{figure}

\lipsum[20]

\section{Primeira seção}

A tabela \ref{Tab:Exemplo} serve para termos também a lista de tabelas. \lipsum[20]

\begin{table}[htb]

  \centering
  \ABNTEXfontereduzida

  \caption{Descrição da pontuação para o critério `custo'}

  \label{Tab:Custo}

  \begin{tabular}{p{2cm}|p{10cm}}

    \textbf{Pontuação} & \textbf{Descrição} \\

    \hline

    $\star$ &
    Custo muito alto, quase impraticável \\

    $\star\star$ &
    Custo alto, considerável \\

    $\star\star\star$ &
    Custo moderado \\

    $\star\star\star\star$ &
    Custo baixo, considerável \\

    $\star\star\star\star\star$ &
    Custo muito baixo, irrisório  \\

    \hline

  \end{tabular}

\end{table}

\lipsum[20]

\section{Segunda seção}

\lipsum[50]
\lipsum[20]

\chapter*{Conclusão}

\lipsum[50]
\lipsum[50]

\postextual             % elementos pós textuais

% Referências bibliográficas

\bibliography{referencias}

% Glossário
%
% Consulte o manual da classe abntex2 para orientações sobre o glossário.
%
%\glossary

% Apêndices
%
% \begin{apendicesenv}
%   \partapendices        % imprime uma página indicando o início dos apêndices
%
%   \chapter{Quisque libero justo}
%
%   \lipsum[50]
%
%   \chapter{Nullam elementum urna vel imperdiet sodales elit ipsum pharetra ligula
%   ac pretium ante justo a nulla curabitur tristique arcu eu metus}
%
%   \lipsum[55-57]
%
% \end{apendicesenv}


% Anexos
%
% \begin{anexosenv}
%
%   \partanexos         % imprime uma página indicando o início dos anexos
%
%   \chapter{Morbi ultrices rutrum lorem.}
%
%   \lipsum[30]
%
%   \chapter{Cras non urna sed feugiat cum sociis natoque penatibus et magnis dis
%   parturient montes nascetur ridiculus mus}
%
%   \lipsum[31]
%
%   \chapter{Fusce facilisis lacinia dui}
%
%   \lipsum[32]
%
% \end{anexosenv}

% Índice remissivo

\printindex

\end{document}
